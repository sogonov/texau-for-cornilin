\begin{sloppypar} % помогает в кириллическом документе выровнять текст по краям
\newpage % Так добавляется  новая страница

\section{Начало работы с Visual Studio} %Объявили начало раздела

В данной лабораторной работе рассматриваются вопросы составления и отладки программ на языке ассемблера, а также совместного использования ассемблера и Си.
Для начала работы необходимо создать директорию с проектом, в которой будут размещены необходимые файлы с текстом исходной программы, служебные файлы и, наконец, будет сформирован исполняемый файл. Были произведены все шаги, описанные в методических указаниях, после чего были исследованы предложенные примеры программ, результаты описаны далее. 
\subsection{Создание исходных файлов} %Объявили начало раздела
Для начала, был создан файл с исходным кодом  на языке Си следующего содержания:
\inputminted[
% frame=lines,%линия сверху и снизу блока кода
% framesep=15mm, % отступ между линией и кодом
baselinestretch=1, %интервал междустрочный
% bgcolor=LightGray, %цвет фона
fontsize=\footnotesize, %размер шрифта
% linenos%нумерация строк
]
{C}%язык программирования
{source1.cpp}%файл с кодом(должен лежать в папке проекта)

Наша программа с помощью функции sprintf записывает в переменную Out текст «Моя программа» и выводит этот текст на экран с помощью функции MessageBox. Результат выполнения на рисунке \ref{ris:Figures/source1.png}.
\imghh{160.5mm}{Figures/source1.png}{Результат выполнения первой программы} 
\subsection{Добавление ассемблерной вставки} %Объявили начало раздела

Следующим этапом было добавление вставки в программу функции на языке ассемблера. Код принимает следующий вид:
\inputminted[
% frame=lines,%линия сверху и снизу блока кода
% framesep=15mm, % отступ между линией и кодом
baselinestretch=1, %интервал междустрочный
% bgcolor=LightGray, %цвет фона
fontsize=\footnotesize, %размер шрифта
% linenos%нумерация строк
]
{C}%язык программирования
{source2.cpp}%файл с кодом(должен лежать в папке проекта)

Наша программа с помощью функции на языке ассемблера складывает два числа, записывает результат в переменную, которую уже Си выведет на экран с помощью функции MessageBox. Результат выполнения на рисунке \ref{ris:Figures/source2.png}
\imghh{160.5mm}{Figures/source2.png}{Результат выполнения первой программы} 
\subsection{Исследование с помощью disassembly} %Объявили начало раздела
Далее программа была изменена, теперь она представляет собой простейший шифратор текста. 
 Код принимает следующий вид:
\inputminted[
% frame=lines,%линия сверху и снизу блока кода
% framesep=15mm, % отступ между линией и кодом
baselinestretch=1, %интервал междустрочный
% bgcolor=LightGray, %цвет фона
fontsize=\footnotesize, %размер шрифта
% linenos%нумерация строк
]
{C}%язык программирования
{source3.cpp}%файл с кодом(должен лежать в папке проекта)
Результат дизассемблирования на рисунке \ref{ris:Figures/source3.png}
\imghh{160.5mm}{Figures/source3.png}{Результат дизассемблирования} 
\imghh{160.5mm}{Figures/source3_2.png}{Результат выполнения} 

\subsection{Организация взаимодействия между программами на ассемблере и Си}

Кроме использования ассемблерных вставок, можно создавать отдельные модули, написанные на языке ассемблера, и представляющие собой самостоятельные подпрограммы. 
Для этого необходимо создать отдельный файл с расширением .asm, и настроить компилятор. 
В результате действий, описанных в указаниях к лабораторной работе, были получены следующие результаты.

Настроен компилятор для автоматического ассемблирования файлов *.asm при сборке с помощью следующих выражений, как показано на рисунке \ref{ris:Figures/compiler.png}
 \begin{minted} [
 % frame=lines,%линия сверху и снизу блока кода
% framesep=15mm, % отступ между линией и кодом
baselinestretch=1, %интервал междустрочный
% bgcolor=LightGray, %цвет фона
fontsize=\footnotesize, %размер шрифта
% linenos%нумерация строк
]{bat}
 

ml /c /Zi %(FileName).asm --command line

%(FileName).obj --outputs

\end{minted}
\imghh{160.5mm}{Figures/compiler.png}{Настройка компилятора} 

Cоздан файл с кодом на языке ассемблера:
{\renewcommand\fcolorbox[4][]{\textcolor{cyan}{\strut#4}}
\inputminted[
% frame=lines,%линия сверху и снизу блока кода
% framesep=15mm, % отступ между линией и кодом
baselinestretch=1, %интервал междустрочный
% bgcolor=LightGray, %цвет фона
fontsize=\footnotesize, %размер шрифта
% linenos%нумерация строк
]
{gas}%язык программирования
{test.asm}%файл с кодом(должен лежать в папке проекта)
}

И изменен код программы на языке Си
\inputminted[
% frame=lines,%линия сверху и снизу блока кода
% framesep=15mm, % отступ между линией и кодом
baselinestretch=1, %интервал междустрочный
% bgcolor=LightGray, %цвет фона
fontsize=\footnotesize, %размер шрифта
% linenos%нумерация строк
]
{C}%язык программирования
{source4.cpp}%файл с кодом(должен лежать в папке проекта)
Результат выполнения программы на рисунке \ref{ris:Figures/source4.png}
\imghh{160.5mm}{Figures/source4.png}{Результат выполнения программы} 


% создается устройство, реагирующее на нажатие клавиш и затем в зависимости от клавиши увеличивающее или
% уменьшающее значение счетчика. Значение счетчика должно отображаться на семисегментном индикаторе, используя принцип динамической индикации. Структурная схема устройства представлена на рисунке \ref{ris:Figures/2022-10-10_22-59-24.png}. На ней показаны основные соединения блоков, за исключением схемы сброса, необходимой при работе с последовательными схемами.


% \imgh{160.5mm}{Figures/2022-10-10_22-59-24.png}{Структурная схема} 


% При проектировании данного устройства использовался принцип иерархического проектирования, где каждый из модулей описывается отдельным описанием устройства на языке VHDL. 

% Делитель частоты DIVIDER обеспечивает тактовые импульсы для модулей устранения дребезга и системы индикации. Он синхронизируется системным тактовым сигналом (100 МГц), подаваемым внешним осциллятором. 


% Модуль устранения дребезга DEBOUNCE подавляет паразитные импульсы, которые могут быть вызваны механическими контактами (кнопками). Это особая схема, синхронизируемая сигналом с выхода делителя частоты DIVIDER. Кнопки и ползунковые переключатели не являются частью ПЛИС, они реализованы на отладочной плате.  


% На реверсивный счетчик BIDIRECTIONAL COUNTER поступают импульсы от модуля устранения дребезга. Реверсивный счётчик увеличивает или уменьшает значение в зависимости от положения ползункового переключателя.

% FSM модуль представляет собой автомат с конечным числом состояний (на англ. finite state machine – FSM) для переключения индикаторов и манипуляции адресами в бесконечном цикле. Мы используем автомат Мура с четырьмя состояниями (S0, S1, S2, S3) и переключением между ними. В каждом состоянии FSM устанавливает определенный адрес для мультиплексора и включает соответствующий индикатор. 

% MUX(мультиплексор) занимается динамической индикацией: на вход получая данные со счетчика и с модуля FSM. FSM включает один из индикаторов, передавая его номер мультиплексору, и мультиплексор выводит соответствующий разряд числа, полученного со счетчика.


% Модуль hexled преобразует число, выданное мультиплексором, в последовательность бит, дающих на индикаторе соответствующую цифру.



% \section{Создание устройства} %Объявили начало раздела
% \subsection{Создание схемы устройства} %Объявили начало раздела
% По методике, описанной в методических указаниях к лабораторной работе, в программном пакете Vivado v2016.4 была создана схема, объединяющая ранее описанные модули. После того, как все модули созданы, необходимо соединить их в единую схему. Полученная схема изображена на рисунке \ref{ris:Figures/sch.png}


% \imgh{160.5mm}{Figures/sch.png}{Схема устройства} 

% \subsection{Симуляция} %Объявили начало раздела

% Далее был создан файл симуляции «simm» подобно тому, как делалось это в I части лабораторной работы. Для того, чтобы просимулировать поведение устройства при всех 
% {VHDL}%язык программирования
% {SIMM.vhd}%файл с кодом(должен лежать в папке проекта)


% Были получены временные диаграммы, показанные на рисунках \ref{ris:Figures/reset.png}, \ref{ris:Figures/updown1.png},  \ref{ris:Figures/updown0.png}, \ref{ris:Figures/updown.png}.

% \imghh{160.5mm}{Figures/reset.png}{Cброс в начальный момент времени} 
% \imghh{160.5mm}{Figures/updown1.png}{Увеличение значения счетчика, при updown='1'} 
% \imghh{160.5mm}{Figures/updown0.png}{Уменьшение значения счетчика, при updown='0'} 
% \imghh{160.5mm}{Figures/updown.png}{Более наглядная иллюстрация} 



% Таким образом, с помощью вышеописанного файла симуляции была проверена работоспособность устройства. Временные диаграммы показывают, что требуемые функции работают.

% \subsection{Синтез} %Объявили начало раздела
% В этом разделе необходимо задать временные ограничения, создать асинхронные группы, и прочие манипуляции, описанные в методических указаниях, добиваясь отсутствия ошибок, связанных с THS  и TNS. 

% В результате синтеза, после исправления была получена следующая схема и сводка по времени(рисунок \ref{ris:Figures/synth.png})
% Так же на этом этапе необходимо указать внешние порты ввода-вывода для размещения схемы внутри ПЛИС. Сделать это нужно вручную.

% \imgh{160.5mm}{Figures/port.png}{Расположение портов ввода-вывода} 
% \begin{landscape}
% \



% \section{Реализация} %Объявили начало раздела
% Следующий шаг разработки устройства – это реализация.
% После завершения процесса реализации нужно проверить выполнение временных ограничений (Report Timing Summary)(рисунок \ref{ris:Figures/timing.png})
% \imghh{160.5mm}{Figures/timing.png}{временная сводка после реализации} 
% \section{Программирование ПЛИС} %Объявили начало раздела

% Для программирования ПЛИС нужно выбрать Generate Bitstream на левой панели (Flow Navigator), а после завершения процесса открыть Hardware Manager, подключить кабель USB к отладочной плате (разъем PROG) и затем к компьютеру. Включить плату (переключатель POWER на плате), а также проверить правильность установки перемычек JP1 и JP2. 

 
% В окне Hardware Manager необходимо выбрать ''Open Target'', а затем ''Auto Connect'' Если соединение будет успешным, надпись ''unconnected'' сменится на имя платы, появится кнопка ''Program Device''. При нажатиии на нее нужно выбрать отладочную плату, а затем файл с битовым потоком (с расширением
% .bit) -- ''Bitstream file'' . Нажать ''Program''.

% После завершения программирования устройства была произведена проверка его работоспособности, в результате которой было выяснено, что устройство работает корректно. Функции, возложенные на кнопки reset, D\_IN и updown правильно работают.
% \section{ОТВЕТЫ НА ВОПРОСЫ ДЛЯ САМОПРОВЕРКИ} %Объявили начало раздела


% \textit {1. Что такое иерархическое проектирование и в чем его преимущества?}

% Иерархические структуры могут упростить процесс разработки и разделить его между несколькими разработчиками. Несколько членов команды
% могут работать одновременно и независимо над разными частями устройства. Каждая часть может быть отлажена по отдельности и стать частью более сложной схемы.

% \textit {2. Пользуясь рисунком \ref{ris:Figures/2022-10-10_22-59-24.png}, поясните работу разрабатываемого устройства.}


% Данный вопрос был прояснен в тексте работы выше.

% \textit {3. Как работает семисегментный индикатор? Что такое статическая и
% динамическая индикация?}



% Индикатор состоит из семи отдельно управляемых (подсвечиваемых светодиодом) элементов - сегментов. Эти элементы позволяют отобразить любую цифру 0..9, а также некоторые другие символы, например: '-', 'A', 'b', 'C', 'd', 'E', 'F' и другие. Это даёт возможность использовать индикатор для вывода положительных и отрицательных десятичных и шестнадцатеричных чисел и даже текстовых сообщений. Обычно индикатор имеет также восьмой элемент - точку, используемую при отображении чисел с десятичной точкой. 

% При статической индикации выводы индикатора подключены к устройству независимо друг от друга и информация на них выводится постоянно. Этот способ управления проще динамического, но без использования дополнительных элементов, подключить многоразрядный семисегментный индикатор к устройству будет проблематично - может не хватить выводов.

% Динамическая индикация подразумевает поочередное зажигание разрядов индикатора с частотой, не воспринимаемой человеческим глазом. Схема подключения индикатора в этом случае на порядок экономичнее благодаря тому, что одинаковые сегменты разрядов индикатора объединены.

% В нашем случае использовался принцип динамической индикации.

% \textit {4. Для чего необходим модуль устранения дребезга?}

% Дребезг контактов это явление, происходящее в электромеханических коммутационных устройствах и аппаратах, длящееся некоторое время после замыкания электрических контактов. После замыкания происходят многократные неконтролируемые замыкания и размыкания контактов за счёт упругости материалов и деталей контактной системы - некоторое время контакты отскакивают друг от друга при соударениях, размыкая и замыкая электрическую цепь. 

% Модуль устранения дребезга призван избавить дальнейшие цепи от этих многократных замыканий и размыканий, выдавая на выходе чистое нажатие.



% \imghh{160.5mm}{Figures/deb.png}{временная диаграмма модуля устранения дребезга -1}
% \imghh{160.5mm}{Figures/deb1.png}{временная диаграмма модуля устранения дребезга -2}

% \textit {5. Объясните работу делителя частоты. Как рассчитать частоту делителя?}


% Nd - переменная, хранящая в себе количество итераций для конструкции generate. Две другие строки отвечают за расположение подключения выводов Clk\_deb и Clk\_led.


% Если синтезировать описание с такими измененными значениями, получим простейший делитель, который делит на четыре (рисунок \ref{ris:Figures/div2.png}). Причем вывод Clk\_deb делится на 4, а Clk\_led делится на два, так как подключены к выходам разных триггеров.

% \imghh{160.5mm}{Figures/div2.png}{делитель в минимальной комплектации} 


% Снова изменив эти коэфициенты: 


% ------------------------
% generic (Nd : integer := 3);--объявление переменной Nd. 
% -------------------------
% Clk_deb<=V(3);--подключение выходов к необходимой ступени делителя
% Clk_led<=V(1);
% ---------------------
% \end{minted}

% Если синтезировать описание с такими измененными значениями, получим делитель, который делит на восемь (рисунок \ref{ris:Figures/div2.png}). Причем вывод Clk\_deb делится на 8, а Clk\_led все еще делится на два.

% \imghh{160.5mm}{Figures/div3.png}{Clk\_deb делим на 8, Clk\_led делим на 2} 

% Аналогично получаем такую схему

% \imghh{160.5mm}{Figures/div3_2.png}{Clk\_deb делим на 8, Clk\_led делим на 4} 

% Если вернуть значения на изначальные, будет синтезирована такая схема(фрагмент показан на рисунке \ref{ris:Figures/div3_22.png}). Видно, что синтезировано 24 пары триггеров и элементов ''и'', выводы Clk\_deb и Clk\_led подключены соответственно к 18 и 19 ступени.

% \imghh{160.5mm}{Figures/div3_22.png}{Схема деления на $2^{18} $ и $2^{19}$} 


% {VHDL}%язык программирования
% {Rcounter.vhd}%файл с кодом(должен лежать в папке проекта)



% {FSM.vhd}%файл с кодом(должен лежать в папке проекта)

% \textit {8. Посмотрите код в файлах divider.vhd и fsm.vhd и скажите, с какой частотой мерцает каждый отдельный семисегментный индикатор.}

% Модуль fsm.vhd управляет анодами семисегментных индикаторов и мультиплексором, включающих нужные для индикации катоды. Мерцание индикаторов обуславливается именно переключением напряжения на их анодах, то есть, это зависит только от модуля fsm.vhd

% fsm.vhd тактируется сигналом, получаемого из модуля divider.vhd, имеющего частоту 100MHz$/2^{19}$ ≈190,734Hz. Именно с такой частотой переключаются индикаторы.


% \textit {9. Прокомментируйте результаты моделирования (рисунок 10 из методических указаний).}

% В ходе работы были подробно прокомментированы  результаты моделирования. Рисунок 10 из методических указаний свидетельствует лишь о том, что  сброс происходит, аноды индикатора последовательно переключаются, сигнал D\_in не влияет ни на что, так как updown='0'

% \textit {10. Обоснуйте выбор портов ввода/вывода. Почему вы назначали именно эти выводы?}
% \imghh{160.5mm}{Figures/port.png}{Расположение портов ввода-вывода}  
% Эти выводы назначены были в соответствии со схемой выводов, взятой из даташита на отладочную плату(рисунок \ref{ris:Figures/datash.png})

% Конструкцией отладочной платы жестко прописаны порты для тактового сигнала, для катодов и анодов семисегментного индикатора. Переключатель для updown можно было выбрать из 15 доступных на плате, кнопки reset и D\_in можно было выбрать из 5 доступных.

% \imghh{160.5mm}{Figures/datash.png}{Фрагмент даташита на отладочную плату}  

% \textit {11. Опишите процесс подключения отладочной платы к компьютеру и ее
% подготовки к программированию ПЛИС}

% Этот процесс был описан в тексте работы выше.


\end{sloppypar}

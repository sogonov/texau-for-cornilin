% Шаблон для курсовой/диплома с поддержкой кириллицы
% Создатель: Полина Шоленинова
% Дата: Май 2021
% Специально для зрителей моего видео на ютубе, в котором я рассказываю как пользоваться Overleaf и Latex
% Посмотреть видео: <ссылка>
% Символ % используется для комментирования кода. Всё, что закомментировано, не компилируется в PDF-файл 

% Ниже начинается код для создания документа, я кратко прокомментировала что что значит
%%%%%%%%%%% Выбираем тип документа %%%%%%%%%%%%%%%%%%%%
\documentclass[12pt, oneside, a4paper]{article} % в [] указываем размер шрифта основного текста, на одной или двух сторонах листа будет распологаться предполагаемый текст и размер документы. В {} указываем класс документа. В Article на верхушке "иерархии" разделов документа стоит Раздел/Секция. Если нужна поддержка Глав, то используйте {report}.

%%%%%%%%%%%% Вставляем нужные пакеты %%%%%%%%%%%%%%%%%%
\usepackage[T2A]{fontenc} % T1 - кодировка для латинской кодировки шрифта, её можно тоже добавить в [] через запятую. T2A - для кириллицы.
\usepackage[utf8]{inputenc} % поддержка UTF-8 
\usepackage[russian, english]{babel} % указываем какие языки должны поддерживаться документом.
\usepackage{graphicx} % добавляем пакет для использования грифики (чтобы вставлять рисунки, фотографии и пр.)
\usepackage{amsmath} % добавляем поддержку математических символов
\usepackage{url} % добавляем поддержку url-ссылок
\usepackage{natbib} % добавляем менеджер цитирования. Их много. Я привыкла работать с natbib.
\bibliographystyle{abbrvnat} % выбираем стиль библиографии отсюда: https://www.overleaf.com/learn/latex/Natbib_bibliography_styles
\setcitestyle{authoryear, open={(},close={)}} % Определяем стиль цитирования. Указываем, чтобы цитирование в тексте вставлялось в формате (Автор, год). 
\usepackage{multirow} % понадобится для создания таблицы с объединенными строками

\usepackage{hyperref} % пакет для интеграции гиперссылок

% настройки цветовой палитры для гиперссылок. Цвета можно на свой вкус выбрать здесь:  https://www.overleaf.com/learn/latex/Using_colours_in_LaTeX
\hypersetup{
    citecolor=gray, % цвет цитирования
    colorlinks=true, 
    linkcolor=black, % цвет для гиперссылок 
    filecolor=magenta, % цвет для ссылок на файл      
    urlcolor=mauve} % цвет для url-ссылок
    \usepackage{listings}

% можно настроить оттенки конкретных цветов (СОВЕРШЕННО ОПЦИОНАЛЬНО)
\usepackage{color}
\definecolor{dkgreen}{rgb}{0,0.6,0} % каждое значение в последнем {} соответсвует красному, зеленому и синему цветам. Играя с этими значениями, получаем различные оттенки. То есть, чтобы настроить оттенок зеленого dkgreen, я выставила зеленый на 0.6, а красный и синий оставила на 0, соотвественно.
\definecolor{gray}{rgb}{0.3,0.3,0.3}
\definecolor{mauve}{rgb}{0.42,0,0.92}

%%%%%%%%%%%% Начало документа %%%%%%%%%%%%%%%%%%

\begin{document} % обозначает начало документа, ниже этого не должно быть пакетов (packages)

\title{Шаблон для курсовой и диплома} 
\author{Имя Фамилия}
\date{Май 2021}
\maketitle
\thispagestyle{empty} % это убирает нумерацию с первой страницы. NB! Если вы выбрали document class {report}, удалите эту строку!

\newpage 
\tableofcontents

\begin{sloppypar} % помогает в кириллическом документе выровнять текст по краям
\newpage 
\section{Введение}
Для примера я вставила в файл рандомно сгенерированный текст. В нем я показала как начать новый абзац и начать текст с красной строки. Также привела пример цитирования. Цитирование совершенно рандомное и приведенные в качестве примеры публикации/книги не относятся к тексту.
Обратите внимание, что это предложение написано жирным шрифтом. А предыдущее - наклонным. Это сделано с помощью двух простых команд.

\section{Теория}
Соображения высшего порядка, а также дальнейшее развитие различных форм деятельности требует от нас анализа модели развития. Равным образом выбранный нами инновационный путь обеспечивает широкому кругу специалистов участие в формировании новых предложений?\\
Равным образом рамки и место обучения кадров требует определения и уточнения существующих финансовых и административных условий.
Дорогие друзья, социально-экономическое развитие требует от нас анализа экономической целесообразности принимаемых решений? \cite{kistyakovskii} % здесь \citep используется для вставки цитирования в скобках

\subsection{Теория 1}

\begin{figure}[!htb]
	\centering
	\includegraphics[width=\textwidth]{Figures/photo1.jpg}
	\caption{Здесь может быть указана подпись рисунка}\label{fig:photo1}
\end{figure}

% Параметра width задаёт ширину рисунка. В этом случае она равна ширине текста, \textwidth. Вместо \textwidth можно укзать значение от 0.1 до 1. \label нужен для того, чтобы потом можно делать сноску на эту картинку, как здесь:

На Рис.\ref{fig:photo1} изображена аудитория. 

\cite{kistyakovskii} показывает, что курс на социально - ориентированный национальный проект способствует повышению актуальности системы масштабного изменения ряда параметров. Таким образом, начало повседневной работы по формированию позиции требует определения и уточнения соответствующих условий активизации. \cite{landau} %обратите внимание, что \citet вставляет цитирование в текст без скобок, чтобы оно вписывалось в текст
показывает, что начало повседневной работы по формированию позиции в значительной степени обуславливает создание дальнейших направлений развития проекта.\\ % - это способ переноса текста на новую строку
Разнообразный и богатый опыт дальнейшее развитие различных форм деятельности в значительной степени обуславливает создание ключевых компонентов планируемого обновления! Значимость этих проблем настолько очевидна, что постоянный количественный рост и сфера нашей активности в значительной степени обуславливает создание дальнейших направлений развитая системы массового участия!
% способ создания нового абзаца, начинающегося с красной строки - простой отступ нажатием enter

Практический опыт показывает, что рамки и место обучения кадров играет важную роль в формировании позиций, занимаемых участниками в отношении поставленных задач. Разнообразный и богатый опыт реализация намеченного плана развития обеспечивает широкому кругу специалистов участие в формировании всесторонне сбалансированных нововведений.
Соображения высшего порядка, а также начало повседневной работы по...

% для примера, разберем как создать таблицу:
\begin{center} % чтобы таблица располагалась по центру
\begin{tabular}{ |c|c|c|c| } % чтобы столбцы были разделены одиночной линией
\hline
Столбец1 & Столбец2 & Столбец3 \\ %этит параметрами задаем названия каждой колонки через знак & 
\hline
\multirow{3}{5em}{Несколько строк} & Ячейка2 & Ячейка3 \\ 
& Ячейка5 & Ячейка6 \\ % В первых фигурных скобках указываем сколько строк объединить, во вторых {} - ширину столбца. Знак \\ указывает, что мы переходим на следующую строку
& Ячейка8 & Ячейка9 \\ 
\hline % просто обозначение линии. Если нужно разделить строки линиями, то тоже можно исоплользовать \hline
\end{tabular}
\end{center}

Больше о таблицах \href{https://www.overleaf.com/learn/latex/Tables}{тут}.


\section{Методы}
\subsection{Математические формулы}
Хорошо известная теорема Пифагора \(x^2 + y^2 = z^2\) была
доказана недействительной для других показателей.
Это означает, что следующее уравнение не имеет целочисленных решений:
\[ x^n + y^n = z^n \]
Другой способ вставить уравнение в текст такой: $x^2 + y^2 = z^2$. То есть уравнение нужно поместить между двумя знаками "доллара".

\subsubsection{Дроби}
При отображении дробей в строке, например \(\frac{3x}{2}\),
вы можете установить другой стиль отображения:
\( \displaystyle \frac{3x}{2} \).
Это также верно и в обратном направлении
\[ f(x)=\frac{P(x)}{Q(x)} \ \ \textrm{и}
\ \ f(x)=\textstyle\frac{P(x)}{Q(x)} \]

\subsubsection{Интегралы}
Интеграл \(\int_{a}^{b} x^2 dx\) внутри текста.
\medskip
Тот же интеграл на дисплее:
\[
\int_{a}^{b} x^2 \,dx
\]
Чтобы не пугать обилием кода и формул, я просто оставлю ссылку на официальный tutorial по интегралам от Overleaf (\href{https://www.overleaf.com/learn/latex/Integrals,_sums_and_limits#Integrals}{ссылка}).

\subsubsection{Сумма и произведение}
Тоже оставлю \href{https://www.overleaf.com/learn/latex/Integrals,_sums_and_limits#Sums_and_products}{ссылку}.

\subsubsection{Пределы}

Предел \(\lim_{x\to\infty} f(x)\) внутри текста.
Тот же предел на дисплее:
\[
\lim_{x\to\infty} f(x)
\]

\section{Результаты}
По этой ссылке - \url{https://www.overleaf.com/learn/latex/List_of_Greek_letters_and_math_symbols} можно посмотреть математические и греческие символы. \href{https://www.overleaf.com/learn/latex/Operators}{Здесь} - математические операторы.



\section{Обсуждение}
\href{https://www.texlive.info/CTAN/info/lshort/russian/lshortru.pdf}{Ссылка} на не очень краткое введение в Latex, но зато полностью на русском. Предлагаю использовать cntrl+F и искать нужные слова и разделы по оглавлению, а не читать от начала до конца.

\section{Заключение}


\newpage
\bibliographystyle{plain}
\bibliography{bibliography}


\end{sloppypar}
\end{document}

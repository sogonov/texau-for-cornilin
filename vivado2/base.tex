\documentclass[a4paper]{article} %размер бумаги устанавливаем А4
\usepackage[T2A]{fontenc}
% \usepackage[utf8]{inputenc} % Включаем поддержку UTF8
% \usepackage[english,russian]{babel} % используем русский и английский языки с переносами
\usepackage{polyglossia}
\setmainlanguage[babelshorthands=true]{russian} % Язык по-умолчанию русский с поддержкой приятных команд пакета babel
\setotherlanguage{english} % Дополнительный язык = английский (в американской вариации по-умолчанию)
\newfontfamily{\cyrillicfonttt}{Times New Roman}
\newfontfamily\cyrillicfont{Times New Roman}
\usepackage{minted} %пакет для подсветки кода
\usepackage{color}
\usepackage{xcolor} % to access the named colour LightGray
\definecolor{LightGray}{gray}{0.9}


% для отступа в первом абзаце
\usepackage{indentfirst} 
\parindent=1.25cm % длина отступа в абзацах
% для продвинутых списков
\usepackage{enumitem} 
\usepackage{amssymb,amsfonts,amsmath,cite,enumerate,float,indentfirst} %пакеты расширений
\usepackage{graphicx}% для вставки картинок
\usepackage{url} % добавляем поддержку url-ссылок
\usepackage{hyperref} % пакет для интеграции гиперссылок
\usepackage{amsmath} % добавляем поддержку математических символов
\usepackage{multirow} % понадобится для создания таблицы с объединенными строками
\usepackage{pdfpages}% Добавление внешних pdf файлов

\usepackage{tocloft}
\renewcommand{\cftsecfont}{\mdseries}
\renewcommand{\cftsecpagefont}{\mdseries}
\cftsetindents{section}{0em}{2em}
\cftsetindents{subsection}{0em}{3em}
\cftsetindents{subsubsection}{0em}{4em}
\renewcommand\cfttoctitlefont{\hfill\normalsize\mdseries}
\renewcommand\cftaftertoctitle{\hfill\mbox{}}
\renewcommand{\cftsecleader}{\cftdotfill{\cftdotsep}}



\usepackage{fontspec}
\setmainfont{Times New Roman} %шрифт 
\graphicspath{{images/}}%путь к рисункам
\usepackage[14pt]{extsizes} % для того чтобы задать нестандартный 14-ый размер шрифта

%\makeatletter
%\renewcommand{\@biblabel}[1]{#1.} % Заменяем библиографию с квадратных скобок на точку:
%\makeatother


\usepackage[tableposition=top]{caption}
\usepackage{subcaption}
\DeclareCaptionLabelFormat{gostfigure}{Рисунок #2}
\DeclareCaptionLabelFormat{gosttable}{Таблица #2}
\DeclareCaptionLabelSeparator{gost}{~—~}
\captionsetup{labelsep=gost}
\captionsetup[figure]{labelformat=gostfigure}
\captionsetup[table]{labelformat=gosttable}
\renewcommand{\thesubfigure}{\asbuk{subfigure}}


\makeatletter

\renewcommand{\section}{\@startsection{section}{1}{0pt}%
                                {-3.5ex plus -1ex minus -.2ex}%
                                {2.3ex plus .2ex}%
{\centering\hyphenpenalty=10000\normalfont\normalsize\mdseries}}
\renewcommand{\subsection}{\@startsection{subsection}{1}{0pt}%
                                {-3.5ex plus -1ex minus -.2ex}%
                                {2.3ex plus .2ex}%
{\centering\hyphenpenalty=10000\normalfont\normalsize\mdseries}}
\renewcommand{\subsubsection}{\@startsection{subsubsection}{1}{0pt}%
                                {-3.5ex plus -1ex minus -.2ex}%
                                {2.3ex plus .2ex}%
{\centering\hyphenpenalty=10000\normalfont\normalsize\mdseries}}
\makeatother





%\renewcommand{\bibname}{Список использованных источников}
%\addcontentsline{toc}{chapter}{Список использованных источников}
% \usepackage{pdflscape}

\usepackage{setspace}
% выравнивание по ширине
\sloppy



%%% Поля и разметка страницы %%%
\usepackage{pdflscape}  % Для включения альбомных страниц
\usepackage{geometry}   % Для последующего задания полей


\geometry{left=3cm}% левое поле
\geometry{right=1.5cm}% правое поле
\geometry{top=2cm}% верхнее поле
\geometry{bottom=2cm}% нижнее поле

\renewcommand{\theenumi}{\arabic{enumi}}% Меняем везде перечисления на цифра.цифра
\renewcommand{\labelenumi}{\arabic{enumi}}% Меняем везде перечисления на цифра.цифра
\renewcommand{\theenumii}{.\arabic{enumii}}% Меняем везде перечисления на цифра.цифра
\renewcommand{\labelenumii}{\arabic{enumi}.\arabic{enumii}.}% Меняем везде перечисления на цифра.цифра
\renewcommand{\theenumiii}{.\arabic{enumiii}}% Меняем везде перечисления на цифра.цифра
\renewcommand{\labelenumiii}{\arabic{enumi}.\arabic{enumii}.\arabic{enumiii}.}% Меняем везде перечисления на цифра.цифра

\newcommand{\imgh}[3]{\begin{figure}[h]\center{\includegraphics[width=#1]{#2}}\caption{#3}\label{ris:#2}\end{figure}}
\newcommand{\imghh}[3]{\begin{figure}[H]\center{\includegraphics[width=#1]{#2}}\caption{#3}\label{ris:#2}\end{figure}}

\begin{document}

\begin{titlepage}
\newpage
\doublespacing
%\linespread{1.3} % полуторный интервал
%\setlength\parindent{1.25cm}
\begin{center}
ФЕДЕРАЛЬНОЕ ГОСУДАРСТВЕННОЕ АВТОНОМНОЕ\\
ОБРАЗОВАТЕЛЬНОЕ УЧРЕЖДЕНИЕ ВЫСШЕГО ОБРАЗОВАНИЯ\\
«САМАРСКИЙ НАЦИОНАЛЬНЫЙ ИССЛЕДОВАТЕЛЬСКИЙ\\
УНИВЕРСИТЕТ ИМЕНИ АКАДЕМИКА С.П. КОРОЛЕВА»	
 \\
\end{center}

\vspace{5em}

\begin{center}
 Институт информатики и кибернетики \\ 
\end{center}

\begin{center}
Кафедра радиотехники \\ 
\end{center}


\vspace{3em}

\begin{center}
{Отчет по лабораторной работе\\''РАЗРАБОТКА ЦИФРОВЫХ УСТРОЙСТВ НА БАЗЕ ПЛИС''}
\end{center}

\vspace{14em}



\newbox{\lbox}
\savebox{\lbox}{\hbox{Корнилин Д.В.}}
\newlength{\maxl}
\setlength{\maxl}{\wd\lbox}
\hfill\parbox{7cm}{
\hspace*{4cm}\hspace*{-4cm}Студент:\hfill\hbox to\maxl{Cогонов Е.А.\hfill}\\
\hspace*{4cm}\hspace*{-4cm}Преподаватель:\hfill\hbox to\maxl{Корнилин Д.В.\hfill }\\
\hspace*{4cm}\hspace*{-4cm}Группа:\hfill\hbox to\maxl{6364-120304D}\\
}


\vspace{\fill}

\begin{center}
Cамара 2022
\end{center}

\end{titlepage}% это титульный лист

\renewcommand{\contentsname}{СОДЕРЖАНИЕ}

{\renewcommand{\baselinestretch}{1.5} %интервал для содержания
\tableofcontents
       
}
%\input{RefProject-referat}% это описание

%\begin{landscape}
% текст в альбомной ориентации
% (таблица, рисунок, схема и т. п.)
%\end{landscape}
\begin{sloppypar} % помогает в кириллическом документе выровнять текст по краям
\newpage % Так добавляется  новая страница

\section{Задание и исходные данные для расчёта} %Объявили начало раздела
Рассчитать параметры универсального активного фильтра второго порядка с единичным усилением, построить график амплитудно-частотной характеристики (АЧХ), и определить наклон АЧХ за пределами полосы пропускания. Тип частотной характеристики фильтра: полосовой фильтр.


Для полосового фильтра: \\Нижняя граничная частота по уровню –3дБ: \begin{math}f_\textup{1}= 33 \textup{ кГц}\end{math} \\
Верхняя граничная частота по уровню –3дБ: \begin{math}f_\textup{2}= 35.1 \textup{ кГц}\end{math} 
\imghh{160.5mm}{Figures/scopes.png}{Схема для расчета } 

1. Необходимо рассчитать значения сопротивлений резисторов и ёмкостей
конденсаторов для схемы, показанной на рисунке \ref{ris:Figures/scopes.png}.

2. Произвести расчёт графика амплитудно-частотной характеристики
требуемого фильтра и построить график АЧХ в диапазоне частот,
выходящем за пределы полосы пропускания на одну октаву. Для полосового фильтра (ПФ) диапазон частот: \begin{math}\textup{ от }0.5\cdot f_\textup{3дБ} \textup{ до } 2\cdot f_\textup{3дБ}\end{math} 

3. По графику АЧХ определить наклон АЧХ за пределами полосы
пропускания: для ПФ – на участках \begin{math}0.5\cdot f_\textup{1} \textup{ ... } f_\textup{1} \textup{  и  } f_\textup{2}\textup{ ... }2\cdot f_\textup{2} \end{math} 

\section{Расчет} %Объявили начало раздела
Дальнейшие вычисления были проведены в пакете Mathcad.
\newpage
% \includepdf[pages={1,3,5}]{myfile.pdf}
\includepdf[pages={-}]{myfile.pdf}



% \newpage
% \bibliographystyle{plain}
% \bibliography{bibliography}


\end{sloppypar}
% это основной документ
%\input{RefProject-Finish}% заключение
%\input{RefProject-App}% приложение

\end{document}